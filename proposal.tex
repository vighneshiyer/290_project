\documentclass[11pt]{article}

\usepackage{float}
\usepackage{hyperref}
\usepackage{graphicx}
% formatting
\usepackage{verbatim}
\usepackage{moreverb}
\usepackage{minted}
\usepackage{parskip}
\usepackage{amsmath}
\usepackage[listings]{tcolorbox}
\usepackage{enumerate}
\usepackage{amssymb}
\usepackage{tikz}
\usepackage{booktabs}
\usetikzlibrary{arrows,automata, positioning}
\let\verbatiminput=\verbatimtabinput
\def\verbatimtabsize{4\relax}

\tcbset{
texexp/.style={colframe=black, colback=lightgray!15,
         coltitle=white,
         fonttitle=\small\sffamily\bfseries, fontupper=\small, fontlower=\small},
     example/.style 2 args={texexp,
title={Question \thetcbcounter: #1},label={#2}},
}

\newtcolorbox{texexp}[1]{texexp}
\newtcolorbox[auto counter]{texexptitled}[3][]{%
example={#2}{#3},#1}

\setlength{\topmargin}{-0.5in}
\setlength{\textheight}{9in}
\setlength{\oddsidemargin}{0in}
\setlength{\evensidemargin}{0in}
\setlength{\textwidth}{6.5in}

% Useful macros

\newcommand{\note}[1]{{\bf [ NOTE: #1 ]}}
\newcommand{\fixme}[1]{{\bf [ FIXME: #1 ]}}
\newcommand{\wunits}[2]{\mbox{#1\,#2}}
\newcommand{\um}{\mbox{$\mu$m}}
\newcommand{\xum}[1]{\wunits{#1}{\um}}
\newcommand{\by}[2]{\mbox{#1$\times$#2}}
\newcommand{\byby}[3]{\mbox{#1$\times$#2$\times$#3}}


\newenvironment{tightlist}
{\begin{itemize}
 \setlength{\parsep}{0pt}
 \setlength{\itemsep}{-2pt}}
{\end{itemize}}

\newenvironment{titledtightlist}[1]
{\noindent
 ~~\textbf{#1}
 \begin{itemize}
 \setlength{\parsep}{0pt}
 \setlength{\itemsep}{-2pt}}
{\end{itemize}}

% Change spacing before and after section headers

\makeatletter
\renewcommand{\section}
{\@startsection {section}{1}{0pt}
 {-2ex}
 {1ex}
 {\bfseries\Large}}
\makeatother

\makeatletter
\renewcommand{\subsection}
{\@startsection {subsection}{1}{0pt}
 {-1ex}
 {0.5ex}
 {\bfseries\normalsize}}
\makeatother

% Reduce likelihood of a single line at the top/bottom of page
\clubpenalty=2000
\widowpenalty=2000

% Other commands and parameters
\pagestyle{myheadings}
\setlength{\parindent}{0in}
\setlength{\parskip}{10pt}

% Commands for register format figures.
\newcommand{\instbit}[1]{\mbox{\scriptsize #1}}
\newcommand{\instbitrange}[2]{\instbit{#1} \hfill \instbit{#2}}

% Break lines on texttt text on underscores
% See: https://tex.stackexchange.com/questions/315369/how-to-deal-with-bad-line-wrapping-of-texttt
\newcommand*\ttvar[1]{\texttt{\expandafter\dottvar\detokenize{#1}\relax}}
\newcommand*\dottvar[1]{\ifx\relax#1\else
  \expandafter\ifx\string_#1\string_\allowbreak\else#1\fi
  \expandafter\dottvar\fi}

\graphicspath{{./figs/}}
\newcommand{\itwos}{I\textsuperscript{2}S}

\begin{document}

\def\PYZsq{\textquotesingle}
\title{\vspace{-0.4in}\Large \bf EE290C Project Proposal: Performance and Energy Characterization of Gemmini, NVDLA, and BOOM Using State-Sampling from Firesim FPGA Simulation\vspace{-0.1in}}
\author{Vighnesh Iyer, Billy Chau}
\date{}
\maketitle

\newcommand{\headertext}{EE290C Project Proposal}
\markboth{\headertext}{\headertext}
\thispagestyle{empty}

\section{Background and Motivation}
The architecture and VLSI literature has produced a litany of dedicated ML accelerators over the last 3 years.
Most of these accelerators exploited unique dataflows, weight/activation sparsity, integer arithmetic, novel circuit techniques, or many other techniques to achieve a claimed perf/watt advantage over prior work.

However, architecture papers typically measure the energy consumption of the proposed accelerator against a baseline using a (Python/C++) model of their accelerator, and the energy numbers may suffer from poor accuracy.
Some papers use HLS to generate RTL implementations of different architectures but the energy calculation from a synthesized netlist and RTL simulation is limited by the speed of simulation, and thus only small networks can be evaluated at the RTL-power-estimation-level.

On the other hand, chip/VLSI papers measure the energy consumption of their accelerator empirically using the taped out chip, but do not have a reference accelerator implementation taped-out to compare against.
Furthermore, once a chip is taped out, the power consumption cannot be measured for individual parts of the accelerator (at module-granularity) and cycle-by-cycle energy numbers cannot be extracted.

\section{Prior Work}

\section{Proposed Extension Over Prior Work}

\section{Project Infrastructure}

\section{Project Timeline}

\begin{enumerate}
  \item \textbf{Checkpoint 1 (4/10)}:
  \item \textbf{Checkpoint 2 (4/24)}:
  \item \textbf{Final Report (5/8)}:
\end{enumerate}

\end{document}
